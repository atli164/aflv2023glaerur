\documentclass{beamer}
\usefonttheme[onlymath]{serif}
\usepackage[T1]{fontenc}
\usepackage[utf8]{inputenc}
\usepackage[english, icelandic]{babel}
\usepackage{amsmath}
\usepackage{amssymb}
\usepackage{amsthm}
\usepackage{gensymb}
\usepackage{parskip}
\usepackage{mathtools}
\usepackage{listings}
\usepackage{hyperref}
\usepackage{graphicx}
\usepackage{color}
\usepackage{enumerate}
\usepackage{verbatim}
\usepackage{minted}
\parskip 0pt


\DeclareMathOperator{\lcm}{lcm}
\newcommand\floor[1]{\left\lfloor#1\right\rfloor}
\newcommand\ceil[1]{\left\lceil#1\right\rceil}
\newcommand\abs[1]{\left|#1\right|}
\newcommand\p[1]{\left(#1\right)}
\newcommand\sqp[1]{\left[#1\right]}
\newcommand\cp[1]{\left\{#1\right\}}
\newcommand\norm[1]{\left\lVert#1\right\rVert}
\renewcommand\Im{\operatorname{Im}}
\renewcommand\Re{\operatorname{Re}}

\usetheme{metropolis}
\definecolor{dark yellow}{rgb} {0.6,0.6,0.0}
\definecolor{dark green}{rgb} {0.0,0.6,0.0}

\graphicspath{{myndir/}}

\title{Introduction}
\author{Arnar Bjarni Arnarson}
\institute{\href{http://ru.is/td}{School of Computer Science} \\[2pt] \href{http://ru.is}{Reykjavík University}}
\titlegraphic{\hfill\includegraphics[height=0.6cm]{kattis}}
\date{\textbf{Árangursrík forritun og lausn verkefna}}

\begin{document}

\begin{frame}[plain]
    \titlepage
\end{frame}

\section*{Sliding Window}

\begin{frame}[plain]
	\frametitle{A Sum Problem}
	\begin{block}{Problem description}
    	    Write a program that, given an integer array of size $N$, finds the
            contiguous subarray of size $K$ with the highest sum.
    \end{block}

    \vspace{10pt}
    
    \begin{block}{Input description}
            Input consist of two lines.
            The first line contains two space separated integers $N$, the size of the array, where $1 \leq N \leq 10^6$,
    and $K$, the size of the subarrays to consider, where $1 \leq K \leq N$.
            Then second line contains $N$ space separated integers, the values of the array.
            Each value in the array is between $-10^9$  and $10^9$.
    \end{block}

    \vspace{10pt}
    
    \begin{block}{Output description}
            Output one line, the sum of the highest valued contiguous subarray of size $K$.
    \end{block}
\end{frame}

\begin{frame}[plain]
	\frametitle{A Sum Problem}
	\begin{center}
		\begin{tabular}{|l|l|}
            \hline
            {\footnotesize Sample input} & {\footnotesize Sample output} \\
            \hline
            \ttfamily
            10 4 & 39 \\
            17 20 0 1 5 24 8 2 4 1 &  \\
            \hline
        \end{tabular}
    \end{center}
\end{frame}

\begin{frame}[plain, fragile]
    \frametitle{Straightforward Solution}
	\begin{scriptsize}
        \begin{minted}{python}
n, k = map(int, input())
arr = list(map(int, input()))
highest = float('inf')
for start in range(n-k+1):
    end = start + k
    total = 0
    for i in range(start, end):
        total += arr[i]
    highest = max((highest, total))
print(highest)
        \end{minted}
    \end{scriptsize}
    \begin{itemize}
        \item<2-> This solution constructs all size $K$ contiguous subarrays.
        \item<3-> What is the time complexity?
        \item<4-> There are $N$ starting points, each construction takes $K$ steps, so $\mathcal{O}(NK)$.
        \item<5-> Too slow!
    \end{itemize}
\end{frame}

\begin{frame}[plain, fragile]
    \frametitle{Wasted Operations}
    \begin{itemize}
        \item<1-> The subarray starting at index $i$ has the sum $a_i + a_{i+1} + \dots + a_{i+k-1}$.
        \item<2-> The subarray starting at index $i+1$ has the sum $a_{i+1} + a_{i+2} + \dots + a_{i+k}$.
        \item<3-> We iterate over the indices $i+1, i+2, \dotsc, i+k-1$ twice.
        \item<4-> What changes between starting at $i$ vs. starting at $i+1$?
        \item<5-> We subtract $a_i$.
        \item<6-> We add $a_i+k$.
        \item<7-> A shift from the subarray starting at $i$ to the subarray starting at $i+1$ takes $\mathcal{O}(1)$ time then.
        \item<8-> This is known as the sliding window technique.
    \end{itemize}
\end{frame}

\begin{frame}[plain, fragile]
    \frametitle{Sliding Window Solution}
	\begin{scriptsize}
        \begin{minted}{python}
n, k = map(int, input())
arr = list(map(int, input()))
total = 0
for i in range(k):
    total += arr[i]
highest = total
for i in range(n - k):
    total -= arr[i]
    total += arr[i+k]
    highest = max((highest, total))
print(highest)
        \end{minted}
    \end{scriptsize}
    \begin{itemize}
        \item<2-> What is the time complexity?
        \item<3-> This solution constructs the first size $K$ contiguous subarray.
        \item<4-> Then, $N-K$ times, an element is removed and another added.
        \item<5-> Subtracting and adding numbers is constant time so $\mathcal{O}(N)$.
        \item<6-> Fast enough!
    \end{itemize}
\end{frame}


\end{document}

