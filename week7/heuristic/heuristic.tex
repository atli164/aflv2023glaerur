\documentclass{beamer}
\usefonttheme[onlymath]{serif}
\usepackage[T1]{fontenc}
\usepackage[utf8]{inputenc}
\usepackage[english, icelandic]{babel}
\usepackage{amsmath}
\usepackage{amssymb}
\usepackage{amsthm}
\usepackage{gensymb}
\usepackage{parskip}
\usepackage{mathtools}
\usepackage{listings}
\usepackage{hyperref}
\usepackage{graphicx}
\usepackage{color}
\usepackage{enumerate}
\usepackage{tikz}
\usetikzlibrary{calc}
\usetikzlibrary{positioning}
\usetikzlibrary{angles}
\usetikzlibrary{shapes}
\usetikzlibrary{arrows}
\usepackage{verbatim}
\usepackage{multicol}
\usepackage{array}
\usepackage{minted}
\parskip 0pt

\usepackage{colortbl}
\usepackage{pgf}
\usepackage{pgfkeys}
\usetikzlibrary{fpu}
\usepackage{tkz-euclide}

\DeclareMathOperator{\lcm}{lcm}
\newcommand\floor[1]{\left\lfloor#1\right\rfloor}
\newcommand\ceil[1]{\left\lceil#1\right\rceil}
\newcommand\abs[1]{\left|#1\right|}
\newcommand\p[1]{\left(#1\right)}
\newcommand\sqp[1]{\left[#1\right]}
\newcommand\cp[1]{\left\{#1\right\}}
\newcommand\norm[1]{\left\lVert#1\right\rVert}
\renewcommand\Im{\operatorname{Im}}
\renewcommand\Re{\operatorname{Re}}

\usetheme{metropolis}
\definecolor{dark yellow}{rgb} {0.6,0.6,0.0}
\definecolor{dark green}{rgb} {0.0,0.6,0.0}

\graphicspath{{myndir/}}

\tikzstyle{vertex}=[circle,fill=black!50,minimum size=15pt,inner sep=0pt, font=\small]
\tikzstyle{selected vertex} = [vertex, fill=red!24]
\tikzstyle{edge} = [draw,thick,-]
\tikzstyle{dedge} = [draw,thick,->]
\tikzstyle{weight} = [font=\scriptsize,pos=0.5]
\tikzstyle{selected edge} = [draw,line width=2pt,-,red!50]
\tikzstyle{selected2 vertex} = [vertex, fill=hilight!50, text=black]
\tikzstyle{ignored edge} = [draw,line width=5pt,-,black!20]
\tikzstyle{vertex1} = [vertex, fill=red]
\tikzstyle{vertex2} = [vertex, fill=blue]
\tikzstyle{vertex3} = [vertex, fill=green, text=black]
\tikzstyle{vertex4} = [vertex, fill=yellow, text=black]
\tikzstyle{vertex5} = [vertex, fill=pink, text=black]
\tikzstyle{vertex6} = [vertex, fill=purple]

\tikzset{
  treenode/.style = {align=center, inner sep=0pt, text centered,
    font=\sffamily},
  vertex/.style = {treenode, circle, black, font=\sffamily\bfseries\tiny, draw=black, text width=1.8em},% arbre rouge noir, noeud noir
  rvertex/.style = {treenode, circle, black, font=\sffamily\bfseries\tiny, draw=red, text width=1.8em},% arbre rouge noir, noeud noir
}

\definecolor{offwhite}{RGB}{249,242,215}
\definecolor{foreground}{HTML}{23373b}
\definecolor{background}{RGB}{24,24,24}
\definecolor{title}{RGB}{107,174,214}
\definecolor{gray}{RGB}{155,155,155}
\definecolor{subtitle}{RGB}{102,255,204}
\definecolor{lolight}{RGB}{155,155,155}
\definecolor{green}{RGB}{125,250,125}

\definecolor{hilight}{RGB}{235,129,27}
\definecolor{vhilight}{HTML}{14B03D}

\def\hepta{\draw[foreground](A) -- (B) -- (C) -- (D) -- (E) -- (F) -- (G) -- cycle;}

\newcommand{\slice}[1]{%
    \hepta
    \draw[foreground] \foreach \x/\y in {#1} {(\x)--(\y)};
}

\title{Heuristics}
\author{Arnar Bjarni Arnarson}
\institute{\href{http://ru.is/td}{School of Computer Science} \\[2pt] \href{http://ru.is}{Reykjavík University}}
\titlegraphic{\hfill\includegraphics[height=0.6cm]{../../shared/kattis}}

\begin{document}
\maketitle

\begin{frame}[plain]{Very Large Graphs}
    \vspace{5pt}
    \begin{itemize}
        \item Consider how you could traverse and search in an extremely large graph.
        \item Specifically, suppose we are trying to reach a specific vertex, our \emph{goal}.
        \item Despite its efficiency, Dijkstra's may prove unsuitable, as it needs to process each vertex and edge.
        \item If $|V| > 10^8$ or $|E| > 10^8$ then the program will be slow.
        \item For example, lets look at \href{https://open.kattis.com/problems/runaheimur}{RúnaHeimur}
    \end{itemize}
\end{frame}

\begin{frame}[plain]{RúnaHeimur}
    \begin{itemize}
        \item The task asks you to solve a sliding puzzle of $N \times M$ tiles.
        \item The state of the puzzle can be represented with a permutation of size $N \times M$, as is done in the statement.
        \item The transitions between states are encoded as swaps of digits.
        \item Solving the problem is simple with a graph algorithm.
        \item DFS or BFS would determine a path to the solved state, if one exists.
        \item Number of states is $\mathcal{O}\left((NM)!\right)$.
        \item Linear time and memory is too much.
    \end{itemize}
\end{frame}

\begin{frame}[plain]{RúnaHeimur}
    \begin{itemize}
        \item Some states are further from the solved state than others.
        \item Need some way of knowing how far we are from the solution.
        \item We can't know for sure, so take a guess.
        \item For a vertex $v$, define a function $h(v)$ to represent the guess
        \item Let $g(v)$ be the distance from the source
        \item Run an algorithm similar to Dijkstra's where vertices are selected by $f(v) = g(v) + h(v)$ instead.
    \end{itemize}
\end{frame}

\begin{frame}[plain]{RúnaHeimur}
    \begin{itemize}
        \item How to define $h(v)$? This is our heuristic function
        \item Number of inversions in the permutation
        \item Number of misplaced tiles
        \item Sum of Manhattan distance to correct positions
        \item Many options available.
        \item Some of these are \emph{admissable} heuristics, meaning they will find the shortest path instead of some path.
    \end{itemize}
\end{frame}

\begin{frame}[plain]{A* and IDA*}
    \begin{itemize}
        \item This method is known as A* and is often used as a simple pathfinding algorithm.
        \item Memory usage can be high.
        \item With a branching factor is $b$ and max depth is $d$ then the complexity is $\mathcal{O}(b^{d}$.
        \item Iterative Deepening DFS is a version of DFS where the depth is bounded. If the algorithm fails then the bound is increased.
        \item This is done to similar performance and results as BFS but with less memory complexity.
        \item Iterative Deppening A* is similar, the bound is on the value of $f(v)$ instead of depth.
    \end{itemize}
\end{frame}

\end{document}
